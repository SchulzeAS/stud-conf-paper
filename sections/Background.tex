\subsection{Java Modeling Language}
JML (Java Modeling Language) is a specification language for Java programs. It allows developers to write formal specifications for Java classes and methods, which can be used to verify the correctness of software through automated testing or formal verification techniques.
JML provides a syntax for writing preconditions, postconditions, and invariants that describe the expected behavior of Java code. Preconditions specify the conditions that must hold before a method is called, while postconditions specify the conditions that must hold after a method completes. Invariants specify the conditions that must hold throughout the execution of a class.
JML annotations can be used to specify the formal specifications of Java classes and methods. These annotations can be used by automated testing tools or formal verification tools to check that a program conforms to its specifications. JML also supports a range of advanced features, including support for object invariants, quantifiers, and model fields.
\subsection{CorC}
CorC is a graphical IDE for developing correct software by using the CbC process (correctness by construction). 
\subsection{JavaPathFinder}
JPF (JavaPathFinder) is a model checker for Java code. Model checkers are a class of testing tools that check all possible paths of execution for a given Java program. 
By exploring these different paths, JPF is able to detect possible defects, such as deadlocks, race conditions, assertion violations or uncaught exceptions
\subsection{JavaPathExplorer}
 The Java PathExplorer (JPAX) can monitor the execution of a Java program and check that it conforms with a set of user provided properties
 formulated in temporal logic. JPAX can in addition analyze the program for concurrency errors such as deadlocks
 and data races.\citep{Havelund_Rosu_2004}
\subsection{KeY prover}
The KeY prover (KeY) is a formal verification tool for Java programs. It allows developers to specify the behavior of Java programs using JML, and then uses automated techniques to verify that the program satisfies its specification.
	